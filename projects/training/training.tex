\documentclass[10pt]{article}
\usepackage{style}


\begin{document}

\pagestyle{fancy}
\fancyhf{}
\fancyhead[L]{Road to Cese}
\fancyhead[C]{Problemi da Proporre}
\fancyhead[R]{\nouppercase{\leftmark}}


\section{Combinatoria}

\subsection{Conteggi}
\begin{enumerate}
    \item \textbf{[9600]} Determinare il numero di quadruple $(x_1,x_2,x_3,x_4)$ di interi positivi dispari tali che $x_1+x_2+x_3+x_4=98$.

    \item \textbf{[0156]} Per quante coppie di interi consecutivi appartenenti all'insieme
        \begin{equation*}
            \{1000,1001,\dots,2000\}
        \end{equation*}
        non è necessario il riporto per svolgere l'addizione fra i due numeri?

    \item \textbf{[2002]} Determinare quanti sono i modi di scegliere cinque numeri dall'insieme $\{1,...,18 \}$ facendo sì che non vengano scelti due numeri consecutivi.

    \item \textbf{[?]} (\textit{AndT-1.2.12}) $\ast$ Siano $p,q$ ed $r$ primi distinti. Definiamo l'insieme
    \begin{equation*}
        S:=\{p^a q^b r^c : 0\le a \le b \le c \le 5\}.
    \end{equation*}
    Determinare il più piccolo intero positivo $n$ tale che ogni sottoinsieme di $S$ di cardinalità $n$ contenga due elementi distinti $x$ e $y$ tali che $x$ divide $y$.

    \item \textbf{[1011]} (\textit{BalticWay-1994}) $\ast$ Un intero positivo è detto goloso se rispetta le seguenti condizioni:
    \begin{enumerate}
        \item le sue cifre appartengono all'insieme $\{ 1,2,3,4,5 \}$;
        \item il valore assoluto della differenza fra due sue cifre consecutive vale sempre 1;
        \item ha esattamente 2024 cifre.
    \end{enumerate}
    Sia $N$ il numero di numeri golosi. Determinare il più grande esponente che compare nella fattorizzazione di $N$.
    
\end{enumerate}

\subsection{Probabilità}
\begin{enumerate}
    \item \textbf{[0027]} Un dado a 10 facce mostra il numero 1 su una faccia, il 2 su due facce, il 3 su tre facce e il 4 sulle quattro facce rimanenti. Il Nick e Foglia tirano ciascuno il dado. Qual è la probabilità che il numero tirato dal Nick sia maggiore del numero tirato da Foglia?\\
    \textit{Dare come risposta la somma di numeratore e denominatore della probabilità ridotta ai minimi termini.}

\end{enumerate}

\subsection{Giochi}
\begin{enumerate}
    \item \textbf{[7003]} Su un tavolo ci sono $n$ monete. Foglia ad ogni mossa può rimuovere dal tavolo un numero di monete pari a 69 o a 104; vince se riesce a rimuovere dal tavolo ogni moneta usando solo queste due mosse, il numero di volte che desidera. Determinare il massimo $n$ che non permette a Foglia di vincere.

    \item \textbf{[?]} Una scuola ha un milione di studenti, fra cui Foglia. Ogni studente ha assegnato un codice segreto di 6 cifre, non necessariamente distinte. L'hacker Tito può entrare nell'account di un qualsiasi studente, una volta soltanto, e copiare $n$ cifre del relativo codice segreto. Le $n$ cifre che copia da account differenti non devono necessariamente essere nelle stesse $n$ posizioni. Determinare il minimo valore di $n$ che permette a Tito di determinare il codice segreto di Foglia.
\end{enumerate}

\subsection{Grafi}
\begin{enumerate}
    \item \textbf{[0007]} (\textit{OmoS.2016-6}) Ad un torneo di basket partecipano 20 squadre. Durante lo svolginmento del torneo, ogni squadra gioca con ogni altra squadra esattamente una volta. Determinare il massimo numero di squadre che, dopo la fine del torneo, possono aver vinto almeno 16 partite.
\end{enumerate}


\newpage
\section{Teoria dei numeri}

\subsection{Algebra}
\begin{enumerate}
    \item \textbf{[0012]} Siano $x$, $y$ e $z$ tre numeri reali tali che $x+y+z=20$ e $x+2y+3z=16$. Determinare il valore di $x+3y+5z$.

    \item \textbf{[0224]} In un negozio, con 10€ si possono comprare 12 penne, con 15€ se ne possono comprare 20. Quante penne si possono comprare, al massimo, con 175€

    \item \textbf{[0054]} (\textit{OmoS.2016-4}) Sia $x$ un numero reale, determinare il minimo valore che l'spressione
    \begin{equation*}
        \lvert x+1 \rvert+3\lvert x+3 \rvert +6\lvert x+6 \rvert +10\lvert x+10\rvert
    \end{equation*}
    può assumere.

    \item \textbf{[0640]} (\textit{OmoS.2016-5}) Una linea $l$ con coefficiente angolare negativo passa per il punto $(20,16)$. Determinare quanto vale al minimo l'area del triangolo compreso fra l'asse-$x$, l'asse-$y$ e la linea $l$.
\end{enumerate}

\subsection{Divisibilità, divisori e fattoriali}
\begin{enumerate}
    \item \textbf{[0008]} Per quanti numeri interi $\alpha$ si ha che
    \begin{equation*}
        \frac{\alpha^3-3\alpha+2}{2\alpha+1}
    \end{equation*}
    è un numero intero?

    \item \textbf{[0083]} Determinare il più grande intero $\alpha$ tale che $23^{6+\alpha}$ divida $2000!$.

    \item \textbf{[0589]} Sia $n:=2^{31}3^{19}$. Quanti sono i divisori positivi di $n^2$ che, pur essendo minori di $n$, non dividono $n$?

    \item Quanti sono gli interi $1\le n \le 7021$ tali che la somma delle cifre di $n$ è multiplo di 7?

    \item \textbf{[1022]} Quanti sono gli interi positivi $n<10000$ tali che il numero ottenuto cancellando l'ultima cifra di $n$ sia un divisore di $n$?

    \item \textbf{[6356]} (\textit{J29T-336}) Consideriamo un intero positivo $n$. Sia $d(n)$ il più grande divisore dispari di $n$. Definiamo
    \begin{equation*}
        D(n):=d(1)+\dots+d(n).
    \end{equation*}
    Determinare la somma dei valori di $n<10000$ tali che $3D(n)=2(1+\dots+n)$.

    \item \textbf{[0210]} (\textit{AndT-1.1.4}) $\ast$ Determinare la somma di tutti gli interi positivi $n$ tali che per ogni intero positivo dispari $a$, se $a^2\le n$, allora $a|n$.

    \item \textbf{[0083]} Determinare il più grande intero $\alpha$ tale che $23^{6+\alpha}$ divida $2000!$.

    \item \textbf{[0028]} (\textit{OmoS.2016-7}) Determinare il numero di quaterne ordinate di interi positivi $(a,b,c,d)$ tali che
    \begin{equation*}
        a!\cdot b!\cdot c!\cdot d!=24!.
    \end{equation*}
    
\end{enumerate}

\subsection{Cifre e congruenze}
\begin{enumerate}
    \item \textbf{[0981]} Sia $1,3,4,9,10,12,13\dots$ la sequenza strettamente crescente formata da tutti i numeri esprimibili come somme di potenze di tre. Determinare il 100-esimo termine di tale sequenza.

    \item \textbf{[0184]} Sia $S:=\{9^0, 9^1,\dots, 9^{4000} \}$. Sapendo che $9^{4000}$ ha 3817 cifre e che la sua prima cifra da sinistra è un 9, determinare quanti degli elementi di $S$ hanno come prima cifra (da sinistra) un 9.

    \item \textbf{[?]} (\textit{J29T-947}) Consideriamo un intero positivo $n$. La somma
    \begin{equation*}
        1+\frac{1}{2}+\dots+\frac{1}{n}
    \end{equation*}
    può essere scritta come $A_n/B_n$, dove $A_n$ e $B_n$ sono primi fra loro. Determinare la somma di tutti gli $n$ tali che $A_n$ sia divisibile per 3.

    \item \textbf{[0024]} (\textit{OmoS.2016-9}) Definiamo $f(x):=1\times 3\times 5\times \cdots \times (2n-1)$. Determinare il resto che l'espressione $f(1)+f(2)+\cdots+f(2024)$ dà nella divisione per 100.
    
\end{enumerate}

\subsection{Diofantee numeri primi}
\begin{enumerate}
    \item \textbf{[2500]} (\textit{AndT-1.2.11}) Definiamo
    \begin{equation*}
        S_a:=\{ n^4+a : n\in \mathbf{N} \}.
    \end{equation*}
    Determinare il maggior intero positivo $4k\le 10000$ tale che $S_k$ non contenga numeri primi.

    \item \textbf{[0013]} Determinare la somma di tutti gli interi $n$ tali che $\sqrt{(4n-2):(n+5)}$ sia un numero razionale.

    \item \textbf{[0295]} Un intero $k$ è detto bello se esiste un primo $p$ minore di 20 tale che $\sqrt {k^2-pk}$ sia un intero positivo. Determinare la somma di tutti i numeri belli.

    \item \textbf{[0102]} (\textit{J29T-384}) Un intero positivo $n$ è detto interessante se è il prodotto di due numeri primi, non necessariamente distinti. Sia $k$ il massimo numero di interi interessanti e consecutivi. Determinare quanto vale, al minimo, la somma di $k$ numeri consecutivi e interessanti.
\end{enumerate}

\subsection{Giochi e altro}
\begin{enumerate}
    \item \textbf{[0225]} Foglia scrive sulla lavagna 10 interi consecutivi. Dopo qualche minuto, arriva il Nick e ne cancella uno. Sapendo che la somma dei nove numeri rimasti è pari a 2020, che numero ha cancellato il Nick?

    \item \textbf{[0011]} (\textit{Nimo}-\textit{Omo}?) Foglia scrive sulla lavagna i numeri da 1 a 10000. Il Nick, ogni minuto, compie le seguenti mosse:
    \begin{enumerate}
        \item sottrae uno ad ogni numero scritto sulla lavagna;
        \item cancella tutti i numeri che, in base dieci, hanno due cifre uguali;
        \item cancella tutti i numeri non positivi.
    \end{enumerate}
    Dopo quanti minuti il Nick avrà cancellato ogni numero dalla lavagna?
\end{enumerate}


\newpage
\section{Geometria}

Le risposte numeriche di alcuni problemi potrebbero contenere radici ed espressioni algebriche brutte. Se non specificato diversamente, dare come risposta le ultime quattro cifre della parte intera del numero brutto.

\subsection{Fake geometria}
\begin{enumerate}
    \item \textbf{[0004]} Quanto vale al minimo l'area di un rombo circoscritto ad una circonferenza di raggio unitario?
    
    \item \textbf{[0640]} (\textit{OmoS.2016-5}) Una linea $l$ con coefficiente angolare negativo passa per il punto $(20,16)$. Determinare quanto vale al minimo l'area del triangolo compreso fra l'asse-$x$, l'asse-$y$ e la linea $l$.
\end{enumerate}

\subsection{Sintetica}
\begin{enumerate}

\item \textbf{[0008]} Un triangolo $\triangle ABC$ è tale che $AC=6$. Un punto $D$ appartenente al lato $AB$ è tale che $AD=DB=CD=5$. Determinare la lunghezza di $BC$.

\item \textbf{[0022]} Determinare il perimetro di un rettangolo di area 32, inscritto in una circonferenza di raggio 4.

\item \textbf{[4330]} Sia $ABCD$ un quadrato di lato $200$. Siano $P$ e $Q$ punti sul lato $AB$ tali che $AP=QB=50$. Sia $E$ il punto sul perimetro di $ABCD$ che massimizza l'ampiezza dell'angolo $\angle PEQ$. Determinare l'area di $\triangle PEQ$.

\item \textbf{[0083]} Sia $ABCD$ un quadrato di lato 4. Siano $P$ e $Q$ due punti sui lati $BC$ e $CD$, rispettivamente, tali che $PB=DQ=1$. Sia $X$ l'intersezione fra $AQ$ e $DP$. Determinare l'area del triangolo $\triangle PXQ$.\\
\textit{Dare come risposta la somma di numeratore e denominatore della frazione che si ottiene ridotta ai minimi termini.}

\item \textbf{[2134]} (\textit{OmoS.2016-8}) Sia $ABCDEF$ un esagono regolare di lato 3. Siano $X$, $Y$ e $Z$ tre punti, rispettivamente, sui lati $AB$, $CD$ ed $EF$, tali che $AX=CY=EZ=1$. L'area di $\triangle XYZ$, semplificata al massimo, è esprimibile come $\frac{a\sqrt{b}}{c}$. Determinare il valore di $100a+10b+c$.

\item \textbf{[0008]} Sia $\triangle ABC$ un triangolo con $BC=4$ e $AC=5$. Sia $D$ il punto medio di $BC$, sia $E$ il piede dell'altezza uscente da $B$ e sia $F$ l'intersezione fra la bisettrice dell'angolo in $C$ e il lato $AB$. Determinare l'area di $\triangle ABC$.

\item \textbf{[0006]} $\triangle ABC$ è un triangolo equilatero di lato 6. Sia $P$ un punto sull'arco $(AC)$ che non contiene $B$ tale che $AP\cdot PC=10$. Determinare la lunghezza di $BP$.

\item \textbf{[0006]} $\ast$ Tre circonferenze di raggio $23$, $46$ e $69$ sono fra loro esternamente tangenti. Determinare il raggio della quarta circonferenza che le tange tutte e tre e che non le comprende.

\end{enumerate}



\end{document}
