\documentclass{article}
\usepackage{../../style}

\begin{document}

\pagestyle{fancy}
\fancyhf{}
\fancyhead[L]{Francesco Bollini}
\fancyhead[C]{Pagellone}
\fancyhead[R]{\nouppercase{\leftmark}}

\section{Geometria}

\begin{proposition}{Casereccio}{}
Sia $D$ un punto sul lato $BC$ del triangolo $ABC$. Definiamo
$K=(ABC)\cap AD$, $X=(ABD)\cap KB$ e $Y=(ACD)\cap BC$. Dimostrare che $X$, $Y$ e $D$
sono allineati.
\end{proposition}

\section{Teoria dei numeri}

\begin{proposition}{}{}
Determinare tutte le coppie di interi $(a,b)$ tali che $a^3+1=b^2(b+2)$.
\end{proposition}
\begin{proof}
	Notiamo che la relazione proposta può essere riscritta come
	\begin{align*}
		a^3 &= (b+1)^3-(b+1)^2-(b+1), \\
			 &= (b+1)(b^2+b-1).
	\end{align*}
	Sia $d$ il massimo comun divisore di $b+1$ e $b^2+b-1$. Osserviamo che $d$ divide
	$b^2+2b+1$ e $b^2+b-1$, quindi divide anche $(b^2+2b+1)-(b^2+b-1)-(b+1)=1$;
	ovvero $d$ è uguale a $1$.
	Dal momento che $b+1$ e $b^2+b-1$ sono primi fra loro e che il loro prodotto
	è un cubo, possiamo affermare che sia $b+1$ che $b^2+b-1$ sono cubi. Definiamo quindi
	$x^3:=b+1$ e $y^3:=b^2+b-1$. Segue immediatamente che $x^6-x^3-1=y^3$.

	Supponiamo che $x$ sia maggiore di $1$. Allora vale la relazione
	\begin{equation*}
		(x^2-1)^3 < x^6-x^3-1=y^3 < (x^2)^3,
	\end{equation*}
	che è una contraddizione, in quanto non vi può essere alcun cubo compreso
	fra due cubi consecutivi. Supponendo che $x=-\alpha$ sia
	minore di $-1$, otteniamo un'ulteriore contraddizione:
	\begin{equation*}
		(\alpha^2)^3 < \alpha^6 + \alpha^3 -1=y^3 < (\alpha^2+1)^3.
	\end{equation*}

	Possiamo dunque affermare che $x$ appartiene all'insieme $\{ -1,0,1 \}$.
	Come si può facilmente verificare,
	le uniche soluzioni intere dell'equazione iniziale sono quindi le coppie
	$(-1,-2),(-1,0)$ e $(0,-1)$.

\end{proof}

\end{document}







