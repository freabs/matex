\documentclass{article}
\usepackage{../../style}

\begin{document}

\pagestyle{fancy}
\fancyhf{}
\fancyhead[L]{Francesco Bollini}
\fancyhead[C]{Pagellone}
\fancyhead[R]{\nouppercase{\leftmark}}


\section{Geometria}

\textit{Aops link:}
\href{https://artofproblemsolving.com/community/c6h1071762p4663879}
{https://artofproblemsolving.com/community/c6h1071762p4663879}.

\begin{proposition}{APMO 2015/1}{}
	Sia $D$ un punto sul lato $BC$ del tirangolo $ABC$.
	Una linea passante per $D$ interseca la retta $AB$ in $X$
	e la retta $AC$ in $Y$.
	Il circocerchio di $BDX$ interseca il circocerchio $\omega$ di $ABC$ in $Z$,
	oltre che in $B$. Le linee $ZD$ e $ZY$ intersecano nuovamente $\omega$ in $V$
	e in $W$, rispettivamente. Dimostrare che $AB=VW$.
\end{proposition}

Proponiamo due dimostrazioni, una sintetica e una più avanzata,
tratta dal thread Aops.

\paragraph{Soluzione sintetica (by me)}
Notiamo che $KD\parallel WV$, dal momento che $\dang{BKD}=\dang{BZD}=\dang{BWV}$.
Osserviamo che
\begin{equation*}
	\dang{BXD}=-\dang{BKD}=-\dang{BWV}=\dang{BAV}.
\end{equation*}
Ne concludiamo che le rette $AV$ e $XD$ sono parallele.
Sia $L$ l'intersezione fra $WV$ e $XY$. Possiamo afferamre che il quadrilatero
$WYCL$ è ciclico, poiché $\dang{LYC}=\dang{YAV}=\dang{VWC}$.
Ma allora
\begin{equation*}
	\dang{XDK}=-\dang{WLY}=-\dang{WCY}=\dang{WCA}=-\dang{ABW}.
\end{equation*}
Da queste relazioni segue che le rette $XD$ e $BK$ sono parallele.
Quindi anche $AV$ e $BW$ sono parallele, che è quanto volevasi dimostrare.

\paragraph{Soluzione per quadrilateri completi (TelvCohl)}
Osserviamo che $Z$ è il punto di Miquel del quadrilatero completo
$\{ AB, BC, AY, YX \}$. Quindi $Z$ appartiene a $(DYC)$. Possiamo ora
affermare che
\begin{equation*}
	\dang{VBW}=\dang{VZW}=\dang{BCA}=\dang{BVA}.
\end{equation*}
Quindi $AV$ e $BW$ sono parallele, da cui la tesi.

\begin{proposition}{Casereccio}{}
	Sia $D$ un punto sul lato $BC$ del triangolo $ABC$. Definiamo
	$K=(ABC)\cap AD$, $X=(ABD)\cap KB$ e $Y=(ACD)\cap BC$.
	Dimostrare che $X$, $Y$ e $D$
	sono allineati.
\end{proposition}

\vspace{0.5cm}
\textit{Aops link:}
\href{https://artofproblemsolving.com/community/c6h1916652p13135961}
{https://artofproblemsolving.com/community/c6h1916652p13135961}.

\begin{proposition}{Ibero 2019/1 -- WC 2020/G2}{}
	Sia $ABCD$ un trapezio isocele inscirtto nella circonferenza $\Gamma$,
	con $AB\parallel CD$.
	Siano, nell'ordine, $P$ e $Q$ due punti sul segmento $AB$, tali che
	$AP=QB$. Siano $E$ ed $F$ i secondi punti di intersezione fra $CP$ e $CQ$
	con $\Gamma$. Le rette $AB$ e $EF$ si intersecano in $G$. Dimostrare che
	$DG$ tange $\Gamma$.
\end{proposition}

Il problema è stato usato nel 2020 per l'ammissione al Winter Camp.
Proponiamo due risoluzioni:

\paragraph{Soluzione proiettiva (by me)}
Siano $M$ il punto medio di $AB$ e $X:=CM\cap \Gamma$.
Allora il fatto che $M$ sia il punto medio di $AB$ come di $PQ$ implica
che i quadrilateri $AXBD$ e $DEXF$ sono armonici. Ne deduciamo che
$AB$, $EF$ e $\bar{D}$ concorrono in $G$, da cui la tesi.

\paragraph{Soluzione per quadrilateri completi (zuss77)}
Sia $X:=DP\cap \Gamma$. Osserviamo che $XF\parallel DC$.
Notiamo che $D$ è il punto di Miquel del quadrilatero completo
$\{ CE, CF, GF, GQ \}$. Questo implica la ciclicità di $GEPD$.
Si può allora affermare che
\begin{equation*}
	\dang{DPG}=-\dang{GEP}=\dang{CEF}=\dang{CXF}=\dang{XCD},
\end{equation*}
da cui la tesi, per il teorema dell'angolo tangente.

\pagebreak
\section{Teoria dei numeri}

\textit{Aops link:}
\href{https://artofproblemsolving.com/community/c6h82828p474988}
{https://artofproblemsolving.com/community/c6h82828p474988}.

\begin{proposition}{APMO 2004/1}{}
	Determinare tutti gli insiemi $S$ non-vuoti di cardinalità finita
	tali che $\frac{a+b}{(a,b)}$ sia un elemento di $S$, per ogni scelta di
	elementi $a$ e $b$ in $S$, non necessariamente distinti.
\end{proposition}

Approciamo il problema dimostrando il fatto seguente.

\begin{claim*}{}{}
	$S$ non può contenere due elementi coprimi.
\end{claim*}
\begin{proof}
	Supponiamo che $x_1:=a$ e $x_2:=b$
	siano due elementi coprimi appartenenti ad $S$.
	Possiamo allora affermare che anche $x_3:=a+b$ è un elemento di $S$.
	Definiamo la successione
	$\{ x_i \}_{i=1}^{\infty}$ secondo la regola $x_{n+2}=x_{n+1}+x_n$,
	allora $(x_{i-1},x_{i})=1$, e quindi $x_i$ è un elemento di $S$
	per ogni $i$. Questo contraddice l'ipotesi che $S$ abbia cardinalità
	finita, dimostrando che $S$ non può contenere due elementi coprimi.
\end{proof}

Supponiamo ora che $x$ sia un elemento di $S$, allora anche
$\frac{x+x}{(x,x)}=2$ appartiene ad $S$. Ovvero, $2$ appartiene
sempre ad $S$, dal momento che $S$ contiene almeno un elemento.
Supponiamo ora che $S$ contanga, oltre al 2, un altro intero positivo
$2^k\cdot y$, con $y$ dispari. Possiamo affermare che anche
$2^{k-1}y+1$ appartiene ad $S$. Ma questa è una contraddizione
per quanto prima dimostrato,
poiché $2$ e $2^{k-1}y+1$ sono coprimi.\\
L'unico insieme che soddisfa le condizioni del problema
è quindi $\{ 2\}$.

\vspace{0.5cm}
\textit{Aops link:}
\href{https://artofproblemsolving.com/community/c6h407219p2274367}
{https://artofproblemsolving.com/community/c6h407219p2274367}.

\begin{proposition}{}{}
	Supponiamo che $a$, $b$ e $c$ siano interi positvi. Dimostrare che
	$a^2+b+c$, $a+b^2+c$ e $a+b+c^2$ non possono essere contemporanemente
	tre quadrati.
\end{proposition}


\begin{proposition}{IUSS 2020}{}
	Determinare tutte le coppie di interi $(a,b)$ tali che $a^3+1=b^2(b+2)$.
\end{proposition}

Le uniche soluzioni sono le coppie $(-1,-2)$, $(-1,0)$ e $(0,-1)$.
Proponiamo due metodi risolutivi.

\paragraph{Soluzione stupida (by me)}

Notiamo che la relazione proposta può essere riscritta come
\begin{align*}
	a^3 &= (b+1)^3-(b+1)^2-(b+1), \\
		 &= (b+1)(b^2+b-1).
\end{align*}
Sia $d$ il massimo comun divisore di $b+1$ e $b^2+b-1$.
Osserviamo che $d$ divide
$b^2+2b+1$ e $b^2+b-1$, quindi divide anche $(b^2+2b+1)-(b^2+b-1)-(b+1)=1$;
ovvero $d$ è uguale a $1$.
Dal momento che $b+1$ e $b^2+b-1$ sono primi fra loro e che il loro prodotto
è un cubo, possiamo affermare che sia $b+1$ che $b^2+b-1$ sono cubi.
Definiamo quindi
$x^3:=b+1$ e $y^3:=b^2+b-1$. Segue immediatamente che $x^6-x^3-1=y^3$.

Supponiamo che $x$ sia maggiore di $1$. Allora vale la relazione
\begin{equation*}
	(x^2-1)^3 < x^6-x^3-1=y^3 < (x^2)^3,
\end{equation*}
che è una contraddizione, in quanto non vi può essere alcun cubo compreso
fra due cubi consecutivi. Supponendo che $x=-\alpha$ sia
minore di $-1$, otteniamo un'ulteriore contraddizione:
\begin{equation*}
	(\alpha^2)^3 < \alpha^6 + \alpha^3 -1=y^3 < (\alpha^2+1)^3.
\end{equation*}

Possiamo dunque affermare che $x$ appartiene all'insieme $\{ -1,0,1 \}$.
Come si può facilmente verificare,
le uniche soluzioni intere dell'equazione iniziale sono quindi le coppie
$(-1,-2),(-1,0)$ e $(0,-1)$.


\paragraph{Soluzione veloce (Massimiliano Foschi)}

Supponiamo che $b$ sia maggiore di 1. Allora vale
la relazione
\begin{equation*}
	b^3<b^3+2b^2-1=a^3<(b+1)^3,
\end{equation*}
che è una contraddizione, in quanto non vi può essere alcun cubo
compreso fra due cubi consecutivi.
Similmente, se $b$ è minore di $-2$, $b$ non può essere un cubo
perché compresa fra due cubi consecutivi. Questo implica che $b$ appartiene
all'insieme $\{-2,-1,0,1 \}$, da cui si ricavano facilmente le soluzioni del
problema.

\vspace{0.5cm}
\textit{Aops link:}
\href{https://artofproblemsolving.com/community/c6h1311821p7030145}
{https://artofproblemsolving.com/community/c6h1311821p7030145}.

\begin{proposition}{Ibero 2016/1}{}
	Determinare tutte le quadruple $(p,q,r,k)$ di numeri primi
	che soddisfano la relazione $pq+qr+rp=12k+1$.
\end{proposition}

Analizzando la relazione data modulo 4 e assumendo
che $p$, $q$ ed $r$ siano dispari, si ottiene una contraddizione:
$3=1 \pmod{4}$. Supponiamo wlog che $r$ sia uguale a 2.
Dalla relazione $pq+2q+2p=12k+1$ otteniamo quanto segue:
\begin{equation*}
	(p+2)(q+2)=12k+5.
\end{equation*}
Studiando tale relazione modulo 3, deduciamo che almeno uno fra
$p$ e $q$ deve essere congruo a 2 $\pmod{3}$, e che quindi
$q=3$, wlog.
Sostituendo $r=2$ e $q=3$ nell'equazione originale, si ottiene
che $5p+5=12k$; ovvero $k=5$ e $p=11$, dal momento che $k$ è
altresì un numero primo.

Le uniche soluzioni del problema sono quindi le
6 permutazioni della quadrupla $(2, 3, 11; 5)$.

\begin{proposition}{BMO 2009/1}{}
	Risolvere l'equazione $3^x-5^y=z^2$ negli interi positivi.
\end{proposition}

Analizzando l'equazione modulo 4, osserviamo che $x$ deve essere pari.
Poniamo $x:=2a$. Per differenza di quadrati, otteniamo che
$5^x=(3^a+z)(3^a-z)$. Sia $3^a+z$ che $3^a-z$ devono quindi essere
potenze di 5. Supponiamo che $3^a+z=5^h$ e che $3^a-z=5^k$.
Per differenza, si ottiene che $k=1$ e che $2\cdot 3^a=5^h+1$. Notiamo che
per $h>1$,
per il teorema di Zsigmondy, $5^h+1$ è divisibile da almeno un fattore primo
diverso da $2$ o da $3$; otteniamo quindi una contraddizione, in quanto
questi ultimi sono gli unici primi che possono dividere il LHS.
L'unica soluzione è dunque la terna $(2,1,2)$.

\vspace{0.5cm}
\textit{Aops link:}
\href{https://artofproblemsolving.com/community/c6h358602p1958953}
{https://artofproblemsolving.com/community/c6h358602p1958953}.

\begin{proposition}{BMO SH 2010/N2}{}
	Risolvere l'equazione $x^3=2y^2+1$ negli interi positivi.
\end{proposition}
Presentiamo una soluzione bucata decisamente troppo bella per essere vera.

\paragraph{Soluzione bucata (by me)}
È noto che un cubo può essere congruo a $0$ o a $\pm 1$ modulo 7.
Tuttavia, un quadrato può essere congruo a 1, 2 o 4 modulo 7.
Dal momento che $2y^2+1 \pmod{7} \in \{2,3,5 \}$ e che
$x^3 \pmod{7} \in \{ 0,\pm 1\}$, l'equazione non ha soluzioni.

\vspace{0.5cm}
\textit{Aops link:}
\href{https://artofproblemsolving.com/community/c6h2326459p18624212}
{https://artofproblemsolving.com/community/c6h2326459p18624212}.

\begin{proposition}{BMO 2020/2}{}
	Determinare tutte le funzioni
	$f:\mathbb{N^{\times}}\rightarrow\mathbb{N^{\times}}$ tali che, per ogni
	intero positivo $n$, valgono le seguenti affermazioni:
	\begin{enumerate}
		\item $ \sum_{k=1}^{n} f(k) $ è un quadrato perfetto;
		\item $f(n)$ divide $n^3$.
	\end{enumerate}
\end{proposition}

Notiamo che la funzione $f(n):=n^3$ soddisfa le considizioni
del problema, poiché che $\sum_{k=1}^n k^3=(1+\cdots+n)^2$.
Dimostriamo che tale funzione è l'unica soluzione del problema.

Supponiamo induttivamente che per ogni $k$ minore o uguale a $n-1$
$f(k)$ valga esattamente $k^3$. Vogliamo dimostrare che
anche $f(n)$ vale $n^3$. Definiamo la funzione $F(n):=\sum_{k=1}^n f(k)$;
allora vale la relazione
\begin{equation*}
	(1+\cdots+(n-1))^2<F(n-1)+f(n)\le (1+\cdots+n)^2.
\end{equation*}
Poiché $F(n-1)+f(n)$ deve essere un quadrato,
possiamo scrivere $F(n-1)+f(n)$ come $\left(\frac{(n-1)n}{2}+x\right)^2$, dove
$x$ è un intero compreso fra 1 e $n$, estremi inclusi. Quindi

\begin{align*}
	f(n) &=\left( \frac{(n-1)n}{2}+x \right) ^2-F(n-1)
	=\left( \frac{(n-1)n}{2}+x \right)^2 - \left( \frac{(n-1)n}{2} \right)^2 \\
		  &= x(x+n^2-n),
\end{align*}
per qualche intero $x$. Sappiamo che $f(n)$ deve dividere $n^3$ e ne concludiamo
che $x(x+n^2-n)$ deve essere minore o uguale a $n^3$. Notiamo che
$x(x+n^2-n)\le n^3$ se e solo se $(n^2+x)(x-n)\le 0$. Il fattore $n^2+x$ è
sempre positivo; deduciamo quindi che $x-n$ deve essere minore o uguale a 0.
Poiché $x$ è compreso fra $1$ e $n$, $x$ deve necessariamente essere uguale a $n$.


\vspace{0.5cm}
\textit{Aops link:}
\href{https://artofproblemsolving.com/community/c6h1916101p13131612}
{https://artofproblemsolving.com/community/c6h1916101p13131612}.

\begin{proposition}{Ibero 2019/1}{}
	Definiamo $s(n)$ come la somma dei quadrati delle cifre di un intero $n$.
	Per esempio, $s(15)=1^2+5^2=26$. Determinare tutti gli interi $n$ per cui
	$s(n)=n$.
\end{proposition}

Supponiamo che $n$ abbia $k$ cifre. Allora $10^{k-1}\le n=s(n) \le 9^2k$,
da cui si ricava che $k$ è minore di 4.
Supponiamo che $n=(abc)$ abbia tre cifre;
allora $s(n)\le 3\cdot 9^2=243$, quindi $a=1,2$. Se $a\le 2$, allora
$a^2+b^2+c^2\le 4+2\cdot 9^2=166$, quindi $a=1$. Questo implica che
\begin{equation*}
	0\le 10b-b^2+99=c^2-c \le 72,
\end{equation*}
che è una contraddizione dal momento che $10b-b^2$ è sempre positivo.
Supponiamo invece che $n=(ab)$ abbia due cifre. Similmente, si ottiene che
\begin{equation*}
	0\le 100a-a^2=b^2-b \le 72,
\end{equation*}
che è una contraddizione. Rimane da analizzare il caso in cui $n=(a)$ ha
una cifra soltanto. Si verifica facilmente che l'unica soluzione, di tale caso
e del problema, è $n=1$.

\end{document}













