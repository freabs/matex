\documentclass{article}
\usepackage{../../../style}

\begin{document}

\pagestyle{fancy}
\fancyhf{}
\fancyhead[L]{Geometria}
\fancyhead[C]{Dispende Olimat}
\fancyhead[R]{\nouppercase{\leftmark}}

\section{Geometria Proiettiva}
In aggiunta alle tecniche presentate in precedenza, proponiamo
ora alcuni elementi di geometria proiettiva che possono tornare utili
nella risoluzione di problemi olimpici. La geometria proiettiva fornisce
infatti una serie di strumenti di notevole efficacia nel risolvere
problemi incentrati sull'incidenza. Spesso problemi che trattano tangenze,
linee parallele o intersezioni possono essere risolti grazie a
tali strumenti.

\subsection{Il piano proiettivo}
La geometria proiettiva non è propriamente una branca della geometria
euclidea, sebbene possa essere usata per risolvere problemi di
costruzione euclidea; essa si fonda infatti su un'estensione del piano
euclideo: il piano proiettivo.\\
Il piano proiettivo, in aggiunta ai punti del piano euclideo, contiene
alcuni \textit{punti all'infinito}, uno per ogni classe di rette parallele.
L'insieme dei punti all'infinito costituisce una retta,
la \textit{retta all'infinito}.
Per visualizzare questa stramba costruzione, immaginiamo essere su una
piazza di forma quadrata, ricoperta di grandi piastrelle anch'esse quadrate.
Guardando la piazza dall'alto verso il basso, mantenendo lo sguardo
perpendicolare al terreno, osserveremo qualcosa di simile a quanto
riportato in figura.

\begin{asy}
	\draw((0,0)--(0,4));
\end{asy}

\end{document}
