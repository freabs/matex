\documentclass{article}
\usepackage{../../../../style}

\begin{document}

\pagestyle{fancy}
\fancyhf{}
\fancyhead[L]{Francesco Bollini}
\fancyhead[C]{Humpty Dumpty}
\fancyhead[R]{12/10/2023}

\textbf{Humpty:}
Il punto Humpty di un tirangolo $ABC$ rispetto al vertice $A$ è
definito come l'intersezione fra la circonferenza $\omega_B$ passante per $B$
che tange $AC$ in $A$ e la circonferenza $\omega_C$ passante per $C$
che tange $AB$ in $C$.

\paragraph{Proprietà di Humpty:}
Detto $P$ il punto $A$-Humpty del triangolo $ABC$ di ortocentro $H$,
valgono le seguenti affermazioni:
\begin{enumerate}
	\item[--] $P$ appartiene alla $A$-mediana;
	\item[--] $P$ giace su $(CHB)$;
	\item[--] $P$ appartiene a $(AHP)$;
	\item[--] $P$ è la proiezione di $H$ sulla $A$-mediana di $ABC$;
	\item[--] Il simmetrico di $P$ rispetto a $BC$ è il punto di intersezione
		fra $S_A$ e $(ABC)$.
\end{enumerate}

\textbf{Dumpty:}
Il punto Dumpty di un triangolo $ABC$ rispetto al vertice $A$ è
definito come il punto di intersezione fra la circonferenza $\omega_B$
che passa per $B$ e tange $AC$ in $A$ e la circonferenza $\omega_C$
che passa per $C$ e tange $AB$ in $A$.

\paragraph{Proprietà di Dupty:}
Detto $P$ il punto $A$-Dumpty de; triangolo $ABC$ di circocentro $O$,
valgono le seguenti affermazioni:
\begin{enumerate}
	\item[--] $P$ appartiene ad $S_A$;
	\item[--] $P$ giace su $(BOC)$;
	\item[--] $P$ appartiene ad $(AO)$;
	\item[--] $P$ è la proiezione di $O$ su $S_A$.
\end{enumerate}

\paragraph{Problemi su Dumpty:}
\begin{enumerate}
	\item Sia $\Gamma$ la circoscritta a triangolo $ABC$.
		La circonferenza $\Omega_A$ è la $A$-mistilinea di $ABC$.
		Sia $P$ il punto di tangenza fra $\Gamma$ e $\Omega_A$ e sia
		$T$ l'intersezione fra una retta parallela a $BC$ che tange
		$\Omega_A$ ed $\Omega_A$. Dimostrare che $\dang{PAQ}=\dang{QAT}$.
	\item (BMO 2022/1) Sia $ABC$ un triangolo con $CA\ne CB$ di circocentro $O$
		e circocerchio $\Gamma$. Siano $t_a$ e $t_B$ le tangenti a $\Gamma$
		in $A$ e in $B$ rispettivamente; sia $Y=t_A\cap t_B$; sia
		$X$ la proiezione di $O$ su $CY$; sia $Z$
		l'intersezione fra $t_A$ e la parallela ad $AB$ passante per $C$.
		Dimostrare che $ZX$ interseca $CA$ nel suo punto medio.
\end{enumerate}

\end{document}
