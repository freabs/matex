\documentclass{article}
\usepackage{../../../style}

\begin{document}

\pagestyle{fancy}
\fancyhf{}
\fancyhead[L]{Teoria dei numeri}
\fancyhead[C]{Dispende Olimat}
\fancyhead[R]{\nouppercase{\leftmark}}

\section{Funzioni moltiplicative}

Ricordiamo che una funzione $f$ è detta $\emph{moltiplicativa}$
se $f(mn)=f(m)f(n)$ per ogni coppia $(m,n)$ di interi coprimi.

\begin{theorem}{}{mult_summation}
	Sia $f$ una funzione moltiplicativa. La funzione		
	$F(n):=\sum_{d \vert n}f(d)$ è anch'essa moltiplicativa.
\end{theorem}
\begin{proof}
	Notiamo che un qualsiasi divisore di $mn$, dal momento che
	$m$ ed $n$ sono coprimi,
	può essere scritto come prodotto di un divisore di $m$ e un divisore di $n$.
	Si ottiene dunque che
	\begin{equation*}
		F(nm)=\sum_{d\vert mn}d=\sum_{d\vert m}d\cdot \sum_{d\vert n}d=F(m)F(n),
	\end{equation*}
	da cui la tesi.
\end{proof}

Applicando il teorema $\ref{theorem:mult_summation}$ alle funzioni moltiplicative
$\textbf{1}(d):=1$ e $\text{Id}(d):=d$, si dimostra senza ulteriori sforzi che
le funzioni $d(n)=\sum_{d\vert n}\textbf{1}(d)$
e $\sigma(n)=\sum_{d\vert n}\text{Id}(d)$, di uso comune nella teoria dei numeri,
sono moltiplicative.
Definiamo ora alcune funzioni che saranno d'aiuto in seguito.

\begin{definition}
La funzione di Möbius $\mu(n)$ è definita come segue:
\begin{equation*}
	\mu(n):=
	\begin{cases}
		1 & \text{se $n=1$,} \\
		(-1)^m & \text{se $n$ non ha divisori quadrati e ha $m$ divisori,} \\
		0 & \text{se $n$ ha almeno un divisore quadrato.}
	\end{cases}
\end{equation*}
\end{definition}

\begin{definition}
La funzione di Dirichlet $\delta(n)$ è definita come segue:
\begin{equation*}
	\delta(n):=
	\begin{cases}
		1 & \text{se $n=1$,} \\
		0 & \text{se $n\ne 1$.}
	\end{cases}
\end{equation*}
\end{definition}

\begin{intheorem}{}{mobmu}
	Per ogni numero naturale $n$, vale la relazione $\sum_{d\vert n}\mu(d)=\delta(n)$.
\end{intheorem}

La dimostrazione è lasciata come esercizio.


\subsection{Convolutione di Dirichlet}
Nell'ambito della teoria dei numeri, la sommatoria $\sum_{d\vert n}$
ricorre in svariate circostanze. La convoluzione è uno strumento
che permette di inquadrare in un contesto formale le varie ricorrenze
particolari, per poterle quindi affrontare con un arsenale teorico
non indifferente.

\begin{definition}
Siano $f$ e $g$ due funzioni aritmetiche, non necessariamente moltiplicative.
La convoluzione di Dirichelet $f\ast g$ delle due funzioni è definita come segue:
\begin{equation*}
	f\ast g:=\sum_{d\vert n}f(d)g\left(\frac{n}{d}\right).
\end{equation*}
\end{definition}

Per fare qualche esempio, possiamo quindi affermare che
$d=\textbf{1}\ast\textbf{1}$, $\sigma=\text{Id}\ast\textbf{1}$,
$\delta=\mu\ast\textbf{1}$ (Teorema \ref{intheorem:mobmu}) e
$\text{Id}=\varphi\ast\textbf{1}$, dove l'ultimo esempio è derivato
direttamente da formule note sulla funzione $\varphi$ d'Eulero.

\begin{exercise}
Verificare che l'operazione $\ast$ è sia commutativa che associativa.
\end{exercise}


\subsection{Inversione di Möbius}

Il teorema di inversione di Möbius, data una funzione aritmetica $f$, permette
di trovare la funzione $g$ tale che $f=g\ast\textbf{1}$, di trovare quindi
la funzione $g$ tale che $\sum_{d\vert n}g(d)=f(n)$. Determinare tale funzione
permette di risolvere immantinente una classe, senza dubbio ristretta, di problemi.

\begin{theorem}{Formula d'inversione di Möbius}{mobinv}
Siano $f$ e $g$ funzioni aritmetiche. Allora, per ogni naturale $n$,
\begin{equation*}
	f(n)=\sum_{d\vert n}g(d) \iff g(n)=\sum_{d\vert n}f(n)\mu(n).
\end{equation*}
\end{theorem}

\begin{proof}
	Ricordiamo che, per il Teorema \ref{intheorem:mobmu} vale la relazione
	$\mu\ast\textbf{1}=\delta$.
	Dimostriamo innanzitutto che $g=f\ast\textbf{1}$, allora $g=f\ast \mu$.
	Osserviamo che se $f=g\ast\textbf{1}$, allora
	$f\ast\mu=(g\ast\textbf{1})\ast\mu=g\ast(\textbf{1}\ast\mu)=g\ast\delta=g$.
	Similmente, se $g=f\ast\mu$, si ha che
	$g\ast\textbf{1}=f\ast(\mu\ast\textbf{1})=f\ast\delta=f$.\\ Questo è quanto
	volevasi dimostrare.
\end{proof}

L'utilità del Teorema \ref{theorem:mobinv} è efficacemente illustrata dal
seguente, classico, esempio.

\begin{example}{Folklore}{}
	Una stringa binaria è detta \emph{goduriosa} se non può essere
	scritta come concatenazione di stringhe identiche più corte. Per esempio,
	la stringa 100100 non è goduriosa, metre 10010 lo è. Determinare il numero
	di stringhe goduriose di lunghezza $n$.
\end{example}
\begin{proof}
	Sia $f(n)$ il numero di stringhe goduriose di lunghezza $n$.
	Notiamo che le stringhe binarie di lunghezza $n$ sono esattamente $2^n$.
	Notiamo inoltre che una qualsiasi stringa di lunghezza $n$ può essere scritta
	in un solo modo come concatenazione di stringhe goduriose
	di lunghezza $d$, dove $d$ è un divisore di $n$. Questo implica che
	\begin{equation*}
		2^n=f\ast\textbf{1}=\sum_{d\vert n}f(d).
	\end{equation*}
	Applicando la formula di inversione di Möbius, otteniamo che
	\begin{equation*}
		f=2^n\ast\mu=\sum_{d\vert n}2^d\cdot\mu\left(\frac{n}{d}\right)
	\end{equation*}
	risolvendodunque il problema.
\end{proof}

\end{document}









