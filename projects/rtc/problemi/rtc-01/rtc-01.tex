\documentclass{article}
\usepackage{../../../../style}

\begin{document}

\pagestyle{fancy}
\fancyhf{}
\fancyhead[L]{RTC 1}
\fancyhead[C]{Qualche Problema}
\fancyhead[R]{10/10/2023}

\textbf{Combinatoria:}
\begin{enumerate}
	\item Quanti sono i modi di disporre le
		lettere $AAABB$ in modo che la stringa $AB$ compaia almeno una volta?
	\item Alla finale di Cesenatico partecipano 32 squadre. In quanti modi
		potrà essere composto il podio?
	\item Quest'anno alla finale di Cesenatico partecipano 32 squadre.
		Le tre squadre che meglio si classificheranno
		andranno in Giappone per rappresentare l'Italia alle IMO.
		Quante sono le possibili delegazioni che gareggeranno alle IMO?
	\item Determinare il numero di quadruple	$(x_1,x_2,x_3,x_4)$
		di interi positivi dispari tali che $x_1+x_2+x_3+x_4=98$.
\end{enumerate}

\textbf{Gemetria}
\begin{enumerate}

	\item I lati $AB$ e $BC$ del rettangolo $ABCD$ misurano
		rispettivamente $6$ e $3$ unità cosmiche.
		Il punto $M$ sul lato $AB$ è tale che $\angle{AMD}=\angle{CMD}$.
		Determinare la misura in gradi di $\angle{AMD}$.

	\item Il punto $P$ giace all'interno del triangolo $ABC$.
		Sappiamo che $\dang{PBA}=20^\circ$ e che $\dang{ACP}=15^\circ$.
		Determinare il valore di in gradi di $\dang{BPC}-\dang{BAC}$.
	
	\item Il punto $D$ appartiene al lato $BC$
		del triangolo $ABC$. Sapendo che $AC=CD$ e che
		$\dang{CAB}-\dang{ABC}=30^{\circ}$,
		determinare la misura in gradi di $\dang{BAD}$.

	\item Il triangolo $ABC$ è isoscele su base $BC$. Le bisettrici
		degli angoli in $B$ e in $C$ si incontrano in $I$.
		Sapendo che $\dang{CIA}=130^\circ$, determinare
		il valore in gradi di $\dang{CAB}$.
\end{enumerate}

\textbf{Teoria dei numeri e Algebra:}
\begin{enumerate}
	\item Per quanti interi
		$n$ il numero $\frac{n^2-6}{n-6}$ è intero?
	\item Per quanti numeri interi  $n$ si
		ha che $\frac{n^3-3n+2}{2n+1}$ è un numero intero?
	\item Determinare il più grande intero $n$
		tale che $n+10$ divida $n^3+100$.
	\item Determinare il numero di coppie di interi positivi $(a,b)$
		che soddisfano le seguenti condizioni:
		\begin{enumerate}
			\item[--] sia $a$ che $b$ sono minori di 2023;
			\item[--] $a$ è maggiore o uguale di $b$;
			\item[--] $a$ divide $2a+b$.
		\end{enumerate}
\end{enumerate}

\end{document}
