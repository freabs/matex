\documentclass{article}
\usepackage{../../../../style}

\begin{document}

\pagestyle{fancy}
\fancyhead[L]{RTC Soluzioni 1}
\fancyhead[C]{Soluzioni}
\fancyhead[R]{10/10/2023}

\textbf{Combinatoria:}
\begin{enumerate}
	\item \textbf{[0009]} Ogni disposizione rispetta le condizioni del
		problema tranne la disposizione $BBAAA$,
		quindi il numero cercato è pari a $\binom{5}{2}-1=9$.

	\item \textbf{[9760]} Poiché alla finale partecipano
		32 squadre, il primo posto può potenzialmente essere assegnato
		a 32 squadre diverse; il secondo posto potrà essere asseganto
		a 31 squadre (tutte tranne quella che è arrivata prima); analogamente,
		il terzo posto può essere assegnato a 30 squadre (tutte tranne la
		prima e la seconda classificata). I podi possibili sono quindi
		$32\cdot 31\cdot 30=29760$. La soluzione del problema è quindi 9760.
	
	\item \textbf{[4960]} Abbiamo visto che i possibili podi sono 29760.
		Consideriamo una delegazione formata dalle squadre $A$, $B$ e $C$.
		Tale delegazione può corrispondere ai sei podi $ABC$, $ACB$, $BAC$, $BCA$,
		$CAB$ e $CBA$ (dove l'ordine delle lettere corrisponde al posizionamento
		delle tre squadre).
		Quindi il numero delle delegazioni corrsponde al numero
		dei podi fratto 6: la soluzione del problema è quindi 4960.

	\item \textbf{[9600]} Poiché $x_i$ è un intero positivo
		dispari, possiamo scriverlo
		nella forma $2y_i-1$ per un qualche intero positivo $y_i$. Ma allora
		$x_1+x_2+x_3+x_4=2(y_1+y_2+y_3+y_4)-4=98$. Il numero di quaterne
		che soddisfano le condizioni del problema corrisponde quindi
		al numero di quaterne $(y_1,y_2,y_3,y_4)$ tali che
		$y_1+y_2+y_3+y_4=51$. Usando il metodo dei separatori, si ricava
		facilmente che questo numero è pari a $\binom{50}{3}$. La soluzione
		del problema è quindi 9600.
\end{enumerate}


\textbf{Geometria:}
\begin{enumerate}
	\item \textbf{[0075]} Osserviamo che
		$\alpha:=\dang{DMA}=\dang{MDC}=\dang{CMD}$,
		quindi $M$ è il punto su $AB$ tale che $DC=CM$. Osserviamo che
		$\dang{BMC}=30^{\circ}$, poiché $CM=2CB$ e $\dang{CBM}=90^\circ$.
		Questo implica che $2\alpha+30^\circ=180^\circ$ e quindi che
		$\alpha=\dang{DMA}=75^\circ$.
	\item \textbf{[0035]} Sia $X$ l'intersezione fra $AP$ e $BC$.
		Allora
		\begin{align*}
			\dang{BPC}-\dang{BAC}&=(\dang{BPX}+\dang{XPC})-(\dang{BAP}+\dang{PAC}) \\
										&=20^\circ+\dang{BAP}+15^\circ+
										\dang{PAC}-(\dang{BAP}+\dang{PAC}=35^\circ.
		\end{align*}

	\item \textbf{[0015]} Indichiamo con $x$ il valore degli angoli
		$\dang{DAC}=\dang{CDA}$. Osserviamo che valgono le relazioni
		\begin{equation*}
		\begin{cases}
				2x=\alpha+\beta, \\
				\alpha-\beta=30^\circ.
			\end{cases}
		\end{equation*}
		Quindi $\alpha=x+15^\circ$ e $\dang{BAD}=\alpha-x=15^\circ$.

	\item \textbf{[0020]} Sia $X$ l'intersezione fra $AC$ e $BI$.
		Per il teorema dell'angolo esterno, possiamo affermare che
		\begin{align*}
			\dang{CIA}&=\dang{CIX}+\dang{XIA}
						=\frac{1}{2}\gamma+\frac{1}{2}\beta+
						\frac{1}{2}\alpha+\frac{1}{2}\beta \\
						 &=\frac{1}{2}\alpha+\frac{3}{2}\beta=130^\circ,
		\end{align*}
		dove l'ultimo passaggio è giustificato dal fattoche $\gamma=\beta$.
		Sappiamo anche che $\alpha+2\beta=180^\circ$. Risolviamo quindi
		il sistema
		\begin{equation*}
			\begin{cases}
				\alpha+3\beta=260^\circ \\
				\alpha+2\beta=180^\circ,
			\end{cases}
		\end{equation*}
		per ottenere che $\alpha=20^\circ$.
\end{enumerate}


\textbf{Teoria dei numeri e Algebra:}
\begin{enumerate}
	\item	\textbf{[0016]} La condizione del provlama è equivalente
		al dire che $n-6$ divide $n^2-6$, ovvero che $n-6\vert n^2-6$.
		Osserviamo che se $n-6\vert n^2-6$, allora $n-6\vert n^2-6
		-(n-6)^2=12n-42$. Quindi $n-6\vert 12n-42-12(n-6)=30$.
		Questo implica che $n-6$ è un divisore non necessariamente
		positivo di $30$. I divisori di 30 sono 16, quindi l'espressione
		data dal problema è intera per 16 valori di $n$.

	\item \textbf{[0008]} La condizione del problema
		è equivalete alla condizione
		$2n+1\vert n^3-3n+2$. Osserviamo che se $2n+1\vert n^3-3n+2$,
		allora $2n+1\vert 8(n^3-3n+2)-(2n+1)^3=-12n^2-3n+15$.
		Ma quindi $2n+1\vert -12n^2-3n+15+3(2n+1)^2=9n+18$. Possiamo ancora
		affermare che $2n+1\vert 2(9n+18)-9(2n+1)=27$. Questo implica che
		$2n+1$ deve essere un divisore, non necessariamente positivo, di 27.
		I divisori positivi o negativi di 27 sono 8, quindi l'espressione
		data dal problema è intera per 8 valori di $n$.
	\item \textbf{[0890]} Supponiamo che $n+10\vert n^3+100$.
		Allora $n+10\vert n^3+100-(n+10)^3=-30n^2-300n-900$, e quindi
		$n+10\vert -30n^2-300n-900+30(n+10)^2=300n+2100$. Allora
		$n+10\vert 300n+2100-300(n+10)=-900$. Possiamo  quindi
		affermare che $n+10$ divide $-900$. Il più grande divisore di $-900$
		è 900, quindi $n$ vale al massimo $890$.
	\item \textbf{[2022]} Sia $(a,b)$ una coppia di interi positivi
		che soddisfa le condizioni del problema.
		Possiamo affermare che $a\vert 2a+b$ e quindi che $a\vert b$.
		Dall'ultima relazione segue che $a\le b$. Ma poiché la coppia
		$(a,b)$ soddisfa le condizioni del problema, si deve avere che
		$a\ge b$; quindi $a$ e $b$ devono essere uguali.
		Il numero di coppie $(a,b)$ con $0<a=b<2023$ sono esattamente
		2022, che è quindi la soluzione del problema.
\end{enumerate}

\end{document}
