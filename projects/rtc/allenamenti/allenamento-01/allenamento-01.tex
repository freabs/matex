\documentclass{article}
\usepackage{../../../../style}


\begin{document}

\pagestyle{fancy}
\fancyhf{}
\fancyhead[L]{RTC A1}
\fancyhead[C]{Allenamento}
\fancyhead[R]{25/09/2023}

\begin{enumerate}
	\item Quanti sono gli interi positivi minori o uguali a 2023 tali che
		la loro scrittura in base 10 contiene lo 0?

		\vspace{0.15cm}
		\textbf{(A)} 469 \hspace{0.5cm}
		\textbf{(B)} 471 \hspace{0.5cm}
		\textbf{(C)} 475 \hspace{0.5cm}
		\textbf{(D)} 478 \hspace{0.5cm}
		\textbf{(E)} 481 \hspace{0.5cm}

	\item Ad una festa di istituto
		partecipano 30 persone, 20 delle quali si conoscono
		fra loro; le altre 10 non conoscono nessuno. Nel corso della festa,
		due persone che si conoscono si abbracciano, mentre due persone che
		non si conoscono si stringono la mano. Quante sono state le strette
		di mano alla fine della festa?

		\vspace{0.15cm}
		\textbf{(A)} 240 \hspace{0.5cm}
		\textbf{(B)} 245 \hspace{0.5cm}
		\textbf{(C)} 290 \hspace{0.5cm}
		\textbf{(D)} 480 \hspace{0.5cm}
		\textbf{(E)} 490 \hspace{0.5cm}
		
	\item Foglia e Tito, allenandosi per Cesenatico, fanno il seguente gioco:
		Foglia sceglie un numero reale compreso fra 0 e 2023, mentre
		Tito sceglie un numero reale compreso fra 0 e 4046. Qual è la
		probabilità che il numero di Tito sia più grande del numero
		di Foglia?

		\vspace{0.15cm}
		\textbf{(A)} 1 \hspace{0.5cm}
		\textbf{(B)} $\frac{2}{3}$ \hspace{0.5cm}
		\textbf{(C)} $\frac{3}{4}$ \hspace{0.5cm}
		\textbf{(D)} $\frac{5}{6}$ \hspace{0.5cm}
		\textbf{(E)} $\frac{7}{8}$ \hspace{0.5cm}

	\item Quanti sono i triangoli di area non-nulla i cui vertici
		appartengono ad una griglia regolare $5\times 5$?

		\vspace{0.15cm}
		\textbf{(A)} 2128 \hspace{0.5cm}
		\textbf{(B)} 2148 \hspace{0.5cm}
		\textbf{(C)} 2160 \hspace{0.5cm}
		\textbf{(D)} 2200 \hspace{0.5cm}
		\textbf{(E)} 2300 \hspace{0.5cm}

	\item Due numeri reali non-nulli sono tali che la loro somma
		sia quattro volte il loro prodotto. Quanto vale la somma dei
		loro reciproci?

		\vspace{0.15cm}
		\textbf{(A)} 1 \hspace{0.5cm}
		\textbf{(B)} 2 \hspace{0.5cm}
		\textbf{(C)} 4 \hspace{0.5cm}
		\textbf{(D)} 8 \hspace{0.5cm}
		\textbf{(E)} 12 \hspace{0.5cm}

	\item Rino ha trovato un numero con una proprietà particolare: se
		lo moltiplica per
		3, aggiunge 11 e inverte l'ordine delle cifre del
		numero che ottiene, si ritrova con un numero compreso fra
		71 e 75, estremi inclusi. Qual è il numero che Rino ha trovato?

		\vspace{0.15cm}
		\textbf{(A)} 11 \hspace{0.5cm}
		\textbf{(B)} 12 \hspace{0.5cm}
		\textbf{(C)} 13 \hspace{0.5cm}
		\textbf{(D)} 14 \hspace{0.5cm}
		\textbf{(E)} 15 \hspace{0.5cm}

	\item Due interi positivi distinti $a$ e $b$ sono tali che $a+b+ab=80$.
		Quanto vale il più grande dei due?

		\vspace{0.15cm}
		\textbf{(A)} 15 \hspace{0.5cm}
		\textbf{(B)} 16 \hspace{0.5cm}
		\textbf{(C)} 18 \hspace{0.5cm}
		\textbf{(D)} 25 \hspace{0.5cm}
		\textbf{(E)} 26 \hspace{0.5cm}

	\item Indichiamo con $S(n)$ la somma delle cifre di un qualsiasi intero
		positivo $n$.
		Per esempio, $S(1507)=13$. Esiste un intero $a$
		tale che $S(a)=1274$. Quale, fra quelli sotto, potrebbe essere
		il valore di $S(a+1)$?

		\vspace{0.15cm}
		\textbf{(A)} 1 \hspace{0.5cm}
		\textbf{(B)} 3 \hspace{0.5cm}
		\textbf{(C)} 12 \hspace{0.5cm}
		\textbf{(D)} 1239 \hspace{0.5cm}
		\textbf{(E)} 1265 \hspace{0.5cm}

	
	\item L'angolo $\dang{BAC}$ del triangolo $ABC$ misura $117^\circ$.
		La bisettrice dell'angolo in $B$ interseca il lato $AC$ nel punto $D$.
		Sapendo che i triangoli $ABC$ e $ABD$ sono simili, determinare
		l'ampiezza in gradi di $\dang{ABD}$.

		\vspace{0.15cm}
		\textbf{(A)} $21^\circ$ \hspace{0.5cm}
		\textbf{(B)} $25^\circ$ \hspace{0.5cm}
		\textbf{(C)} $26^\circ$ \hspace{0.5cm}
		\textbf{(D)} $27^\circ$ \hspace{0.5cm}
		\textbf{(E)} $30^\circ$ \hspace{0.5cm}

	\item La bidelleria è un rettangolo $ABCD$ con $AB=3$ e $BC=4$. Sia $E$ la
		proiezione di $B$ su $AC$. Quanto vale l'area del triangolo $ADE$?

		\vspace{0.15cm}
		\textbf{(A)} $1$ \hspace{0.5cm}
		\textbf{(B)} $\frac{42}{25}$ \hspace{0.5cm}
		\textbf{(C)} $\frac{28}{15}$ \hspace{0.5cm}
		\textbf{(D)} $2$ \hspace{0.5cm}
		\textbf{(E)} $\frac{54}{25}$ \hspace{0.5cm}

	\item L'angolo in $A$ del triangolo $ABC$ è ottusangolo. I punti
		$D$ ed $E$ giacciono sul lato $BC$ in modo che $\dang{BAD}=\dang{BCA}$
		e $\dang{CAE}=\dang{CBA}$. Sapendo che $AB=10$, $AC=11$ e che $DE=4$,
		determinare la lunghezza di $BC$.

		\vspace{0.15cm}
		\textbf{(A)} 12 \hspace{0.5cm}
		\textbf{(B)} 16 \hspace{0.5cm}
		\textbf{(C)} 17 \hspace{0.5cm}
		\textbf{(D)} 21 \hspace{0.5cm}
		\textbf{(E)} 13 \hspace{0.5cm}

	\item Il quadrato $ABCD$ ha lato 2. Un semicerchio di diametro $AB$ è
		costruito internamente al quadrato; la tangente al semicerchio
		passante per $C$ interseca $AD$ nel punto $E$.
		Determinare la lunghezza di $CE$.

		\vspace{0.15cm}
		\textbf{(A)} $\frac{3}{2}$ \hspace{0.5cm}
		\textbf{(B)} $\frac{5}{2}$ \hspace{0.5cm}
		\textbf{(C)} $\frac{7}{4}$ \hspace{0.5cm}
		\textbf{(D)} $\frac{9}{4}$ \hspace{0.5cm}
		\textbf{(E)} $\frac{9}{5}$ \hspace{0.5cm}

\end{enumerate}

\end{document}
