\documentclass{article}
\usepackage{../../../../style}

\begin{document}

\pagestyle{fancy}
\fancyhf{}
\fancyhead[L]{RTC A2}
\fancyhead[C]{Allenamento}
\fancyhead[R]{02/10/2023}

\begin{enumerate}

	\item Gli interi fra 1 e 8 sono posizionati ai vertici di un cubo,
		in modo che la somma dei quattro numeri su ogni faccia sia sempre
		uguale. Quanto vale tale somma?

	\item Una sera a Cesenatico, ognuno dei sette membri di una squadra
		può decidere se andare a prendere o un gelato o una piadina, senza
		prenderli entrambi. Sapendo che almeno una persona prende un gelato
		e che almeno una persona prende una piadina, quante sono le possibili
		combinazioni gelato-piadina?

	\item Quanti sono i modi di colorare tre caselle di una griglia $5\times 5$
		in modo che non vi siano due caselle colorate sulla stessa riga o
		sulla stessa colonna?

	\item Venti quadrati di lato 1 sono disposti in una tabella
		$10\times 2$. Due quadrati sono detti adiacenti se hanno in comune
		esattamente 2 vertici. In quanti modi si possono colorare esattamente
		9 di questi quadrati senza colorarne due adicenti?

	\item Supponiamo che $a+a^{-1}=4$ per un qualche numero
		reale $a$. Quanto vale $a^4+a^{-4}$?

	\item Quante sono le coppie di interi positivi $(a,b)$ con
		$a>b$ tali che $a^2-b^2=96$?

	\item Due interi postivi $m$ ed $n$ sono tali che $m^3=75n$.
		Quanto può valere al minimo la somma $m+n$?

	\item Indichiamo con $\mu(n)$ il prodotto degli interi postivi
		che dividono $n$, contando 1 e $n$ stesso.
		Sapendo che $\mu(\mu(\mu(10)))=10^x$, quanto vale $x$?

	\item I lati del pentagono convesso $ABCDE$ sono fra loro congruenti.
		Sapendo che $\dang{A}=\dang{B}=90^\circ$, quanto misura in gradi
		l'angolo $\dang{DEA}$?

	\item Sia $O$ il circocentro del triangolo $ABC$. Sappiamo che
		gli angoli $\dang{BOC}$ e $\dang{AOB}$ valgono rispettivamente
		$120^\circ$ e $140^\circ$ gradi.
		Quanto misura, in gradi, l'angolo $\dang{ABC}$?

	\item I lati del triangolo $ABC$ sono fra loro in rapporto $3:4:5$.
		Sapendo che il raggio della circonferenza circoscritta ad $ABC$
		è pari a 3, è possibile scrivere l'area del triangolo nella forma
		$\frac{a}{b}$, dove $a$ e $b$ sono due interi positivi coprimi.
		Determinare il valore di $a+b$.

	\item Un punto $P$ giace all'interno del triangolo equilatero $ABC$.
		Siano $D$, $E$ ed $F$ i piedi delle altezze condotte da $P$
		ai lati $AB$, $BC$ e $CA$, rispettivamente. Sapendo che $PD=1$,
		$PE=2$ e $PF=3$, è possibile scrivere la lunghezza di $AB$ nella
		forma $a\sqrt{b}$, dove $a$ e $b$ sono due interi positivi e
		$b$ non è divisibile per il quadrato di nessun primo. Determinare
		il valore di $100a+b$.

\end{enumerate}

\end{document}

